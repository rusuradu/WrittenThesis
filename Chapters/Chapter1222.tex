\chapter*{Introduction}
\lhead{Introduction}
\rhead{Radu-George Rusu}
\addcontentsline{toc}{chapter}{Introduction}
Team project based learning has become increasingly popular among universities learning process. It involves creating groups of students that receive a certain task to be solved in a collaborative manner. Beyond the fact that it can be used in all sorts of practical works, on new topics or to deepen already reviewed topics, this type of learning develop social skills which a person encounter in real life, such as teamwork, oral and written communication \cite{peev} and management skills. In engineering degrees, such as computer science, this type of learning can simulate a real work environment matching up any industry software development company or research group working environments.

Using team project based learning you get all the advantages that come with it, but some problems occur. The evaluation process of the final result is hard and tricky to manage. Lecturers have to evaluate every student individually, based on their work inside the group, but when the evaluation process take place, the lecturer see only the final result, and the students that are part of a group are likely to be seen, from the point of view of the lecturer, as a merged entity, and not separate individuals, what makes the final evaluation to be more a group contribution evaluation, rather than individual contribution evaluation. In this manner some students may be favored while others may be disadvantaged. A simple solution for this problem is to include students into evaluation of their colleagues. Although simple, this solution raises other problems: how can you mitigate the social influence that appear when students evaluate each other, such as a student evaluating fellow friend, what methods of grades aggregation to use etc. It is widely known that averaging grades to obtain a final grade is not the best way since, the arithmetic mean is a measure of central tendency, and it is highly influenced by the values tagged as noise (larger or smaller values than most of the values).

Social choice theory is described in \cite{scdef}, as the study of collective decision processes and procedures. It is widely used when you have a group of people tries to reach a common decision. In the evaluation process, assuming that we mitigate the social influences when students evaluate each other, we can use election methods (e.g. Borda Count, Kemeny-Young) in order to reach a common grade, based on the grades received.

This thesis presents a method of evaluation for team project based learning which is trying to mitigate the main problems that occurs in the evaluation process. It will also compare grading methods that are used in different universities, mainly methods which use numerical grades, rankings between students and methods based on quantitative descriptions of desired behaviour. It makes use of the social choices theory tools to get a more suitable way of grades aggregation, and creates a fair environment for every entity which is part of the evaluation process. 

The first chapter, Introducing FiiEv Method, presents methods for evaluation processes that are currently used and comparisons between them, highlights the main advantages and drawbacks for every method, and also presents how it can be merged in a new method, that gets all the advantages and lose some disadvantages. Another aspects discussed in this chapter is about usages of voting protocols can be used in the learning process and grades aggregation methods.

In the second chapter, FiiEv Application, it is described a web application that facilitates the evaluation process for team project based learning for both students and lectures. It will present technical details used in the development of the application, theoretical details and proofs of the methods described in first chapter. Also, it will contains screen-shoots of the user interface with details on how to use it and why you should use this method in the evaluation process.

The third chapter, Experimental Results, shows two ways of testing the method described in chapter 1 and results that were obtained afterwards. It begins with testing the method on some computer generated data: a sets of programs, developed in such a way that simulates real life situation as best as it can, are subject to the evaluation method and the results are collected and analyzed. The second part shows the method apply on real life situation. Three groups of students from the Faculty of Computer Science, were asked to evaluate the work of their colleagues, during development process of the final project of the Software Engineering course. The data collected was also subject to the evaluation method and the results were taken, analyzed and compared with current existing methods.

The next and final chapter, Conclusions, presents the main conclusions that can be made about the team project based learning, the usage of the method described in this paper, and the improvements that can be done over this method.