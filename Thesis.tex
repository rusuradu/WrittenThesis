%% ----------------------------------------------------------------
%% Thesis.tex -- MAIN FILE (the one that you compile with LaTeX)
%% ---------------------------------------------------------------- 

% Set up the document
\documentclass[a4paper, 12pt, oneside]{Thesis}  % Use the "Thesis" style, based on the ECS Thesis style by Steve Gunn
\graphicspath{Figures/}  % Location of the graphics files (set up for graphics to be in PDF format)

% Include any extra LaTeX packages required
\usepackage[square, numbers, comma, sort&compress]{natbib}  % Use the "Natbib" style for the references in the Bibliography
\usepackage{verbatim}  % Needed for the "comment" environment to make LaTeX comments
\usepackage{vector}  % Allows "\bvec{}" and "\buvec{}" for "blackboard" style bold vectors in maths
\hypersetup{urlcolor=blue, colorlinks=true}  % Colours hyperlinks in blue, but this can be distracting if there are many links.
\usepackage{titlesec}
\usepackage{amssymb}
\usepackage{pdflscape}
\usepackage{array}
\usepackage{longtable}
\usepackage[table]{xcolor}
\usepackage{caption}
\usepackage{float}
%% -----------------Front page-------------------------------------
\begin{document}
\pagestyle{empty}
\centerline{\small{'ALEXANDRU IOAN CUZA UNIVERSITY' IA\c SI}}
\vspace{0.5cm}
\centerline{\textbf{\Large{\textsf{FACULTY OF COMPUTER SCIENCE}}}}
\vspace{3cm}
\begin{center}
	\includegraphics[width=3cm,height=3cm]{Pictures/fii-01.png}
\end{center}
\vspace{3cm}
\centerline{\Large{MASTER THESIS}}
\vspace{1cm}
\centerline{\textbf{\LARGE{Expectation - Maximization}}}
\vspace{0.2cm}
\centerline{\textbf{\LARGE{for aerial view ship detection}}}
\vspace{1cm}
\centerline{proposed by}
\vspace{1cm}
\centerline{\textbf{\Large{\textsf{Radu-George Rusu}}}}
\vspace{2cm}
\centerline{\textsf{\textbf{Session:} \textit{July, 2019}}}
\vspace{0.5cm}
\centerline{Scientific coordinator}
\vspace{0.5cm}
\centerline{\textsf{\textbf{\large{???, Dr. Liviu Ciortuz}}}}

\clearpage
%% ----------------------------------------------------------------

%% --------------Page Title----------------------------------------
\pagestyle{empty}
\centerline{\textbf{\large{\textsf{ALEXANDRU IOAN CUZA UNIVERSITY IA\c SI}}}}
\vspace{0.5cm}
\centerline{\textbf{\large{\textsf{FACULTY OF COMPUTER SCIENCE}}}}
\vspace{5cm}
\centerline{\textbf{\LARGE{Expectation - Maximization}}}
\vspace{0.2cm}
\centerline{\textbf{\LARGE{for aerial view ship detection}}}
\vspace{3cm}
\centerline{\textbf{\Large{\textsf{Radu-George Rusu}}}}
\vspace{2cm}
\centerline{\Large{{\textsf{\textbf{Session:} \textit{July, 2019}}}}}
\vspace{4.5cm}
\centerline{\textbf{Scientific coordinator}}
\vspace{0.5cm}
\centerline{\textsf{\textbf{\large{???, Dr. Liviu Ciortuz}}}}

\clearpage
%% ----------------------------------------------------------------

\setstretch{1.3}  % It is better to have smaller font and larger line spacing than the other way round

% Define the page headers using the FancyHdr package and set up for one-sided printing
%\fancyhead{}  % Clears all page headers and footers
%\rhead{\thepage}  % Sets the right side header to show the page number
%\lhead{}  % Clears the left side page header

\pagestyle{fancy}  % Finally, use the "fancy" page style to implement the FancyHdr headers
%% ----------------------------------------------------------------
% Declaration Page required for the Thesis, your institution may give you a different text to place here

%% ----------------------------------------------------------------
% The "Funny Quote Page"
%\pagestyle{empty}  % No headers or footers for the following pages

%\null\vfill
% Now comes the "Funny Quote", written in italics
%\textit{``Write a funny quote here.''}

%\begin{flushright}
%If the quote is taken from someone, their name goes here
%\end{flushright}

%\vfill\vfill\vfill\vfill\vfill\vfill\null
%\clearpage  % Funny Quote page ended, start a new page
%% ----------------------------------------------------------------

% The Abstract Page
%\addtotoc{Abstract}  % Add the "Abstract" page entry to the Contents
%\abstract{
%\addtocontents{toc}{\vspace{1em}}  % Add a gap in the Contents, for aesthetics

%The Thesis Abstract is written here (and usually kept to just %this page). The page is kept centered vertically so can %expand into the blank space above the title too\ldots

%}

%\clearpage  % Abstract ended, start a new page
%% ----------------------------------------------------------------

%\setstretch{1.3}  % Reset the line-spacing to 1.3 for body text (if it has changed)

% The Acknowledgements page, for thanking everyone
%\acknowledgements{
%\addtocontents{toc}{\vspace{1em}}  % Add a gap in the Contents, for aesthetics

%The acknowledgements and the people to thank go here, don't forget to include your project advisor\ldots

%}
%\clearpage  % End of the Acknowledgements
%% ----------------------------------------------------------------

\pagestyle{empty}  %The page style headers have been "empty" all this time, now use the "fancy" headers as defined before to bring them back


%% ----------------------------------------------------------------
%\lhead{An evaluation method for team project based learning}  % Set the left side page header to "Contents"
%\rhead{Radu-George Rusu}
%\cfoot{}
\pagenumbering{gobble}
\tableofcontents  % Write out the Table of Contents

%% ----------------------------------------------------------------
%\lhead{\emph{List of Figures}}  % Set the left side page header to "List if Figures"
%\listoffigures  % Write out the List of Figures

%% ----------------------------------------------------------------
%\lhead{\emph{List of Tables}}  % Set the left side page header to "List of Tables"
%\listoftables  % Write out the List of Tables

\addtocontents{toc}{\vspace{2em}}  % Add a gap in the Contents, for aesthetics


%% ----------------------------------------------------------------
\mainmatter	  % Begin normal, numeric (1,2,3...) page numbering
\pagestyle{fancy}  % Return 
\cfoot{\thepage}
% Include the chapters of the thesis, as separate files
% Just uncomment the lines as you write the chapters

\chapter*{Introduction}
\lhead{Introduction}
\rhead{Radu-George Rusu}
\addcontentsline{toc}{chapter}{Introduction}
Machine Learning (ML) is the field that addresses the problem of how to use historical data to produce accurate models or predictions for new data. A Machine Learning Algorithm, based on data-points from the historical data or train data, produces a model that can be later used for giving the best approximation for new inputs or test data. With the increase in computing resources in the recent years, a lot of ML algorithms have became usable in practice in various situations.

The process of computing the model is called training or learning. Starting from this, the ML learning processes can be split into two main categories: \textit{supervised learning} and \textit{unsupervised learning}. In \textit{supervised} learning the training data set (historical data), has labels, the classification (number of classes and the model of each class) is known at the training phase. After the training, the computed model must classify the new inputs (test data) with the best class approximation of the input. When the unsupervised learning procedure is used, the classes of train data are not known from the beginning (data has no labels), but a certain model is enforced on them (number of classes, a distribution function that generated the data etc.), and the training phase produces the best parameters for the enforced model.

A linked domain with ML is computer vision. Recently, with the incentive for autonomous driving, automatic image/video moderation and others, the computer vision algorithms have become more used in practical environments. Three main problems can be highlighted from this field: \textit{object classification}, \textit{object detection} and \textit{image segmentation}. Object classification is the process of assign a label to a certain image, linked to the object that is inside that picture (i.e. labeling and image with a cat with the label "Cat"). Object detection in images try to find the exact position of an object in an image (usually by giving bounding boxes for every object of interest in the picture). In most practical cases detection is used together with classification. The image segmentation problem refers to grouping pixels that are "semantically similar".

This thesis take into consideration both supervised learning and unsupervised learning, and tries to solve a segmentation problem, in conjunction with object classification.

This thesis is structured to discuss the problem statement that is proposed for solving, the theoretical background needed, and two proposed solutions with the corresponding results and comparison between them.

The Airbus Ship Detection Challenge section \ref{DatasetChapter} describes the origin of the data set used in this thesis, an analysis over how the data looks and how is distributed and a metric that will be used for later evaluation of the models. The theoretical background section \ref{TheoreticalBackground} presents the models and the basic theory behind them, that will be used in the solutions.

The experiment setup section \ref{ExperimentSetup} is showing the thought process of the solution and the two solutions with differences in structure and example runs. The result section \ref{Results} is summarizing the results by the two models, and other base models, and compares them to one another. The last chapter \ref{Conclusions} draws the conclusions of this thesis and sets the future improvements/experiments that can be conducted on the same problem.
 % Variable Selection

\chapter{Airbus Ship Detection Challenge}
\lhead{Airbus Ship Detection Challenge}
\rhead{Radu-George Rusu}
\label{DatasetChapter}

This chapter presents the data set used for experimenting in this paper, it's the Kaggle competition that it was used in, and an analysis over how the data looks and can impact the experiments.

\begin{figure}[H]
	\includegraphics[width=\textwidth]{Pictures/003TrainingSetExamples.png}
	\caption{Training set examples}
	\label{TrainSetExample}
	%\textbf{Figure 2. Hill Climbing algorithm} [15]
\end{figure}

\section{Airbus Ship Dataset}
The Airbus Ship Dataset is an image dataset that was used in the Airbus Ship Detection Challenge \cite{AirbusDataSetChallenge} on \url{www.kaggle.com}. The aim of this competition is to detect ships inside an image taken from an aerial view. As the competition description says, this detection can help multiple organizations such as environmental organization, insurance companies to have knowledge of illegal activities done at sea. A few examples of the training set can be found in Figure \ref{TrainSetExample}.% Todo !!!Challenge with clouds ports etc etc small sizes?


\section{Analysis}
The competition provides both testing and training dataset in a large amount. The train set is composed of \textbf{231723} pictures, of size \textbf{768x768} pixels, in JPEG format, each of them having a set of pixels assigned as being ship pixels. To avoid specifying each pixel separately, for size reasons, the run length encoding is used. Every pixel is numbered from 1 to maximum size, top down, left right order. Pixel (1, 1) will have index 1, pixel (1, 2) index 2 etc. Run length encoding is formed of 2k numbers, the numbers in odd position (1 - indexed) specifies the starting index of a pixel ship sequence, and the ones in even position specifies the length of the sequence.
For example, the sequence: \textit{5 3 10 2}, encodes the ship pixels \textit{\{5, 6, 7, 10, 11\}}. Those encodings are provided via an \textbf{.csv} file, with two columns, \textbf{ImageId} and \textbf{EncodedPixels}. Also, in this file, every ship is given as a separate entry, so a picture can appear multiple time in this file, and at least once. (See table \ref{traindfhead})\\

\begin{table}[H]
	\centering
	\begin{tabular}{|c|c|l|}
		\hline
		ImageId & EncodedPixels \\ \hline
		00003e153.jpg & NaN         \\ \hline
		0001124c7.jpg & NaN         \\ \hline
		000155de5.jpg & 264661 17 265429 33 266197 33 266965 33 267733...         \\ \hline
		000194a2d.jpg & 360486 1 361252 4 362019 5 362785 8 363552 10 ...   \\ \hline
		000194a2d.jpg & 51834 9 52602 9 53370 9 54138 9 54906 9 55674 ...   \\ \hline
		000194a2d.jpg & 198320 10 199088 10 199856 10 200624 10 201392...  \\ \hline
		000194a2d.jpg & 55683 1 56451 1 57219 1 57987 1 58755 1 59523 ...  \\ \hline
		000194a2d.jpg & 254389 9 255157 17 255925 17 256693 17 257461 ...  \\ \hline
		0001b1832.jpg & NaN  \\ \hline
		00021ddc3.jpg & 108287 1 109054 3 109821 4 110588 5 111356 5 1...  \\ \hline
	\end{tabular}
	\captionof{table}{Train data set sample}\label{traindfhead}
\end{table}

The test set provided by the competition contains \textbf{15607} pictures of the same size as the ones in the training set, and the result that should be submitted should also use the run length encoding described above.


On a first inspection of this dataset, it can be observed that there are more picture with no ship in them (0 encoded pixels), then the number of pictures with ships (as shown in table \ref{shipnoshiptable}). It can be seen that for every picture that have at least one ship in it, there are two pictures with no ship in it.

\begin{table}[H]
	\centering
	\begin{tabular}{|c|c|c|l|}
		\hline
		Total & No Ship Pictures & Ship Pictures \\ \hline
		231723 & 150000 & 81723 \\ \hline
	\end{tabular}
	\captionof{table}{Ship/no ship count training data}\label{shipnoshiptable}
\end{table}

Figure \ref{ShipNumberImage} has a histogram of number of images grouped by number of ships inside an image. The number of pictures that have only 1 ship stands out as being around 35\% from the total number of images with ship, and there are also pictures that contains up to 15 ships.

\begin{figure}
	\includegraphics[width=\textwidth]{Pictures/001ShipNumberImage.png}
	\caption{ Number of ships/image}
	\label{ShipNumberImage}
	%\textbf{Figure 2. Hill Climbing algorithm} [15]
\end{figure}

The last metric that should be analyzed in this dataset is the ship size in pixels, to measure the magnitude of the task. As the table \ref{shipzisepixeltable} presents, the average size is around 1500 pixels, with half of the ships in the training set being under 408 pixels in size. Based on this we can draw the conclusion that one ship that should be detected in the test set form less than 0.0006\% of the entire image.

\begin{table}
	\centering
	\begin{tabular}{|c|c|c|c|c|c|l|}
		\hline
		Average & Min & Q1 & Median & Q3 & Max\\ \hline
		1,567.40 & 2.00	& 111 & 408.00 & 1550 & 25,904.00 \\ \hline
	\end{tabular}
	\captionof{table}{Ship size in pixels}\label{shipzisepixeltable}
\end{table}

\section{Evaluation Metric}
To evaluate possible solutions for this competition, the $F_2$ score was used. This is defined as follows:
\begin{itemize}
	\item We define a set of thresholds $T = \{0.5, 0.55, 0.6, \dots 0.95\}$.
	\item For every threshold $t \in T$:
	\item the $IoU$ between predicted pixels (pp) and true pixels (tp) of a ship is computed as $IoU = \frac{pp \cap pt}{pp \cup pt}$
	\item if this value is higher than $t$ then this item is consider a true positive (a hit)
	\item if this value is lower than $t$ then this is consider either a false negative (a predicted ship that has no true ship) or a false positive (there is no predicted ship for a true ship)
	\item the $F_2(t) =\frac{5TP(t)}{5TP(t) +4FN(t) + FP(t)}$ is computed
	\item the final score for an image is computed as $FS(image) = \frac{\sum_{t \in T} F_2(t)}{\big{|}T\big{|}}$
\end{itemize}

It is worth mentioning that this metric, in this form, penalizes more false negatives, than false positives which means that it encourages rather to not predict a ship if the algorithm is not certain of there is a ship in an area. This will have later implications in this paper. For this reason, the "no-machine algorithm" behaves really well, since this will give a percentage of images with no ships over a dataset.

TODO add some images with their mask :D % Data sets

\chapter{Theoretical background}
\lhead{Theoretical background}
\rhead{Radu-George Rusu}
\label{TheoreticalBackground}

This chapter will briefly describe the theoretical models used in the experiments presented in this paper.

\section{Expectation-Maximization}
The Expectation-Maximization (EM) is an algorithmic template for finding the Maximum Likelihood Estimation (MLE) parameters in statistical models which are based on missing or latent data.\\
The Maximum Likelihood Estimation problem can be defined as follows:

\fbox{
\begin{minipage}{\textwidth}
\textbf{MLE Problem}\\
\textbf{Input}\\
Y - set of observed data\\
$p$ - probability distribution that we assume generated the Y data set

\textbf{Output}\\
 $h^{MLE} = \underset{h}{argmax}$  $L(Y | h) = \underset{h}{argmax}$  $p(Y | h) $, where $h$ is the set of distribution parameters
\end{minipage}
}

 When the entire set Y is formed out of observable data, the above problem can be easily solved. A common approach for this instance of the problem, is to take the derivative of the log-likelihood function and solve it for $h$. However, real life data is not always fully observed, and inside of the given data we have latent variables. In this particular case, the above-mentioned method doesn't work anymore, and an EM approach can be used.
 
 In the case of latent data we can consider the Y set as being a reunion between $X \cup Z = Y$, where X is the set of visible data, and Z is the set of hidden/latent data. The initial method doesn't work since it is impossible to compute $p(Z | h)$.
 
 \begin{figure}[H]
 	\includegraphics[width=\textwidth]{Pictures/004EMScheme.png}
 	\caption{EM Algorithm \cite{emCiortuz}}
 	\label{EmScheme}
 	%\textbf{Figure 2. Hill Climbing algorithm} [15]
 \end{figure}
 
 The EM algorithm (as show in \ref{EmScheme}) applies an iterative method, that for each iteration computes two steps:
 \begin{itemize}
 	\item E-Step: expectation of the latent variables, based on the distribution parameters from the previous iteration and the visible data ($\mathbf{E[Z | X, h^{(t)}]}$).
 	\item M-Step: based on the results from the E-Step, the latent variables, can be considered as observed and the $h$ parameters for this step can be solved like in the classic MLE problem.
 \end{itemize}

In the most basic form of this algorithm, the initial values for $h^{(0)}$ are randomly selected, and as a stop condition a maximum number of iterations is  used, but these two conditions can vary from problem to problem.
 
 
\section{Neural Networks for object detection}
In computer vision and object detection/recognition in images, the classical neural networks involves a high number of weights that have to be learned, which in turn, makes the networks work slow on images that are bigger in size and sometimes even impractical. The main advantage of the \textbf{Convolutional Neural Networks (CNN)} \cite{ConvNeuralNetwork} is the idea of using the convolution operator (\ref{Convolution}), which drastically reduces the number of weights to be learned during training phase and makes them usable in practice.

 \begin{figure}[H]
	\includegraphics[width=\textwidth]{Pictures/006Convolution.png}
	\caption{Convolution operator \cite{ConvNeuralNetwork}}
	\label{Convolution}
	%\textbf{Figure 2. Hill Climbing algorithm} [15]
\end{figure}

Other two popular operators are the AveragePooling and MaxPooling, which as the name suggests, just extract either the average or the maximum value of the values in the corresponding square. Although these operators doesn't have as much expression power as the linear combination \ref{Convolution}, they don't have any weights to be learned during the training phase.

The convolution operator result size depends on the following variables:
\begin{itemize}
	\item $n$ - size of the input image (considering the input image is squared)
	\item $f$ - size of the filter (also squared)
	\item $p$ - padding of the initial image
	\item $s$ - stride of the convolution operation
\end{itemize}
The convolution operator result size is:
$$ (n, n) * (f, f) = \bigg( \bigg\lfloor \frac{n + 2 \times p - f}{s} + 1 \bigg\rfloor , \bigg\lfloor \frac{n + 2 \times p - f}{s} + 1 \bigg\rfloor \bigg)$$

To observe the advantages of CNN over the classical NN we can take an example of a network formed only of two layers. A first one of size $1024\times1024$ which can be viewed as the input image, that is given in grayscale, and a second layer of size $512\times512$, which is a first feature extraction layer.

In a classical neural network every input from the first layer must be connected with every output on the second layer, and every connection has it's own weight, which, in the above case, would result in computing $1024 \times 1024 \times 512 \times 512 = 274,877,906,944$ weights. Also, in computer vision problem there is a need of more than one intermediate layer to extract the features inside an image.

In the case of CNNs, the only weights that have to be computed for a layer are the values inside the kernel, which in the above example can be considered a kernel of size $512 \times 512 = 262,144‬$ with a stride of 1 and no padding. The number of weights that will be learned it's way less than in the case of classical neural networks, which allows more layers, and makes the CNN more practical for this problem. Also, in practical cases, the convolution operator can be an average/maximum over the target values, instead of a linear combination, and there will be no weights to learn on that level.

 \begin{figure}[H]
	\includegraphics[width=\textwidth]{Pictures/007SkipConnection.png}
	\caption{Residual Block \cite{ResNetPaper}}
	\label{SkipConnection}
	%\textbf{Figure 2. Hill Climbing algorithm} [15]
\end{figure}

Another issue that occurs in practical object detection in image cases, is the depth of the network. In order to learn complex features inside an image, a neural network needs a high number of layers, to have space for learning every feature, which, in practice, leads to the vanishing gradient problem \cite{ResNetPaper}. This problem makes the training of these networks hard or even impossible, depending on the task. To solve this \textbf{Residual Networks} \cite{ResNetPaper} were developed.

The fundamental idea of these networks is the addition of the residual blocks (\ref{SkipConnection}). In this blocks, the inputs are once again added to the output value of a later layer, before the activation function, avoiding a "squishing" derivative, which in turn will result in a higher overall derivative of the entire block.

With these methods, neural networks structures with 34, 50 or 152 layers were developed and used as show in \ref{ResNetExample}.

 \begin{figure}[H]
	\includegraphics[width=\textwidth]{Pictures/008ResNetExample.png}
	\caption{Residual Networks Structure \cite{ResNetPaper}}
	\label{ResNetExample}
	%\textbf{Figure 2. Hill Climbing algorithm} [15]
\end{figure}

The neural networks described above behave well on object recognition and object detection problems. Another common case in image processing is the segmentation problem, where similar pixels inside a picture must be grouped together. In this particular case, the CNN might not work well in the form described above, because every pixel of the image can have it's own classification, and the convolution operator is specialized in contracting image features, hence reducing the number of outputs of the network. To solve this problem, \textbf{U-Nets} networks \cite{Unet} have been developed. The name U-Net comes from the visual representation of the architecture (\ref{UnetArchitecture}) which resembles the U letter. There are two parts of this architecture, the contracting path, and the expansive path. The contracting path behaves as an usual convolutional neural network extracting features from the image. The expansive path uses the Transposed Convolution Operator \cite{ConvolutionOperatorPaper}, which helps the networks reconstruct the image at the initial resolution, from the feature vector that was computed by the contracting path. 

 \begin{figure}[H]
	\includegraphics[width=\textwidth]{Pictures/012Unet.png}
	\caption{U-Net Architecture \cite{Unet}}
	\label{UnetArchitecture}
	%\textbf{Figure 2. Hill Climbing algorithm} [15]
\end{figure} % Algorithms

\chapter{Experiment setup}
\lhead{Experiment setup}
\rhead{Radu-George Rusu}
\label{ExperimentSetup}

This chapter will present the thought process and the setup that was used to run the experiments on the dataset presented in Chapter \ref{DatasetChapter}.

\section{Initial idea}

Taking a high level look through the training data set, a general case can be seen: pictures are taking at sea (no port in site) and as such, they are also conceived from water and ships (which can divide pixels in two group as shown in image \ref{ShipExampleAtSea}). If the images are moved into grayscale, and its histogram is computed (\ref{ImageHistogram}), then the pixels of the image can be split into two groups, which will also give the ship/no-ship classification. This problem is classical problem that can be solved using EM algorithm.

\begin{figure}[h]
	\centering
	\includegraphics[height=0.2\textheight]{Pictures/002ShipExampleatSea.png}
	\caption{Ships at sea}
	\label{ShipExampleAtSea}
\end{figure}

\begin{figure}[h]
	\centering
	\includegraphics[width=\textwidth]{Pictures/005ImageHistogram.png}
	\caption{Histogram of grayscale at sea image}
	\label{ImageHistogram}
\end{figure}

\subsection{Expectation-Maximization}

\subsubsection{Intuition}


\subsection{Limitations} % Conslusion

\chapter{Results}
\lhead{Results}
\rhead{Radu-George Rusu}
\label{Results}

The Airbus Ship Dataset Challenge \cite{AirbusDataSetChallenge} offers an automatic method of evaluation that, for every solution, computes the $F_2$ score. In order to avoid leader-board probing (competitors trying to over-fit the model on the test data), the test data set is split in two datasets. A public test dataset that contains $12\%$ of the original test images, and a private test dataset that contains the other, $88\%$ of the original test images.

To have a base model for comparison against the presented solutions, a No-Machine algorithm has been chosen. This algorithm evaluates every pixel as being a background pixel (no detection involved). The $F_2$ score for this algorithm is actually a percentage of how many images doesn't have ships in the image.

In the image recognition area is known that neural networks behave well, so, another model was chosen to have a top score value for comparison. This model is a combination between a residual network (ResNet34) which classifies the image in ship/no-ship images, and for images that have ships a U-Net, trained in the same fashion as the U-Net solution presented in this thesis, for the final classification phase. For the rest of the section we will refer to this solution as Hybrid.
\begin{table}[h]
	\centering
	\begin{tabular}{|c|c|c|}
		\hline
		Method & Private Test Dataset & Public Test Dataset \\ \hline
		No-Machine & 0.76566	& 0.52090  \\ \hline
		Region Proposal & 0.78052	& 0.58810 \\ \hline
		U-Net & 0.78424	& 0.61663  \\ \hline
		Hybrid & 0.79396 & 0.62259 \\ \hline
	\end{tabular}
	\captionof{table}{Results on the competition score}
	\label{resultOnComp}
\end{table}

The table \ref{resultOnComp} presents the results as scored by the Kaggle automatic evaluation. Both of the proposed solution have a better score than the No-Machine algorithm, which means that the two proposed solutions correctly identify images that have ships and the ones that does not. The results of the two proposed solutions are similar on both datasets, with small improvements when using the U-Net solution. This improvement is happening because of the granularity that the U-Net is using over the region proposal solution, which uses hard-coded dimensions and positions for the crops that can became region proposal for final prediction. It is worth mentioning that the $F_2$ score, computed as described in \ref{DatasetChapter}, penalizes false negatives over no prediction, which explains the small difference between No-Machine and the proposed solutions. As seen in \ref{resultOnComp}, the two solution have the tendency to classify small pockets of pixels as ships, which will drastically decrease the $F_2$ score for that respective image. Both of the proposed solutions behave worse than the Hybrid solution. The Region Proposal solution has the same weakness of using bulk windows for predictions while the Hybrid solution, as the U-Net proposal, has the advantage of granularity because of the end network used. In the case of U-Net Proposal, due to the pixel exclusion phase, it loses some information from the picture when the pixels are turned black, which leads to a weaker final prediction.

\begin{figure}
	\includegraphics[width=\textwidth]{Pictures/016Comparison4.png}
	\caption{False Negative}
	\label{false_neg_size}
\end{figure}

Analyzing \ref{Final_Result}, the two algorithms, behaves in a similar fashion. The better behavior of the U-Net solution can be spotted since the false negatives produced by this, are smaller in area than the ones from the Region Proposal solution \ref{false_neg_size}. Again, the rigid character of the Region Proposal solution is seen in the \ref{false_neg_size}, where a big part of the port wasn't declassified because it was part of the same region with a ship inside, which the second Residual Network classified it as ship. Although from a visual perspective the U-Net architecture gives a better result, the $F_2$ score will be similar (not equal since the two predictions have different area of the ship mask).

\begin{figure}[H]
	\includegraphics[width=\textwidth]{Pictures/016Comparison1.png}\\
	\includegraphics[width=\textwidth]{Pictures/016Comparison2.png}
	\caption{False Positive}
	\label{false_positive}
\end{figure}

Overall the two solutions are comparable in terms of $F_2$ score, but on a visual analysis of the results, the U-Net architecture produces less false positives \ref{false_positive} and false negatives. The figure \ref{Final_Result} presents a few more uses cases that confirms the above conclusions.

In the figure \ref{Final_Result_Port}, it is shown that pictures with ports are still a challenge for both solutions. The weak flexibility of the region proposal determined by using bulk windows, can be spotted when a ship is presented in the same window with a piece/or multiple pieces of the port, which in turn also leads to a lot of false negatives. The U-Net solution also struggles to differentiate the port from the ships, the results detection having a lot of pockets that will be evaluate as false negatives and in turn a smaller $F_2$ score.

\begin{figure}
	\includegraphics[width=0.9\textwidth]{Pictures/016Comparison3.png}
	\includegraphics[width=0.9\textwidth]{Pictures/016Comparison5.png}
	\includegraphics[width=0.9\textwidth]{Pictures/016Comparison6.png}
	\caption{Final Result comparison}
	\label{Final_Result}
\end{figure}

\begin{figure}
	\includegraphics[width=0.9\textwidth]{Pictures/016Comparison7.png}
	\includegraphics[width=0.9\textwidth]{Pictures/016Comparison8.png}
	\caption{Final Result Port Comparison}
	\label{Final_Result_Port}
\end{figure} % Conclusion

\chapter*{Bibliography}
%\lhead{Introduction}
%\rhead{Radu-George Rusu}
\addcontentsline{toc}{chapter}{Bibliography}
[1] - http://www.statsoft.com/Textbook/Multiple-Regression

[2] - http://www.statisticshowto.com/akaikes-information-criterion/

[3] - http://www.stat.cmu.edu/~cshalizi/mreg/15/lectures/26/lecture-26.pdf

[4] - http://www.stat.columbia.edu/~martin/W2024/R10.pdf

[5] -  https://www.sciencedirect.com/topics/medicine-and-dentistry/akaike-information-criterion

[6] - https://en.wikipedia.org/wiki/Least-angle\textunderscore regression

[7] - https://en.wikipedia.org/wiki/Lasso\textunderscore (statistics)

[8] - https://profs.info.uaic.ro/~pmihaela/MOC/GA.html

[9] - https://www.omicsonline.org/open-access/a-novel-improved-genetic-algorithm-based-on-the-fixed-point-theoremand-triangulation-method-jcsb-1000227.php?aid=77064

[10] - https://www.doc.ic.ac.uk/{$\sim$nd/surprise\textunderscore96/journal/vol1/hmw/article1.html
	
[11] - https://www.researchgate.net/post/Can\textunderscore anyone\textunderscore send\textunderscore me\textunderscore Algorithm\textunderscore for\textunderscore crossover\textunderscore and\textunderscore mutation\textunderscore ie\textunderscore pseudo-code\textunderscore of\textunderscore crossover\textunderscore and\textunderscore mutation

[12] - https://github.com/KendallPark/genetic-algorithm

[13] - https://profs.info.uaic.ro/~pmihaela/MOC/trajectory.html

[14] - https://en.wikipedia.org/wiki/Hill\textunderscore climbing

[15] - https://www.geeksforgeeks.org/introduction-hill-climbing-artificial-intelligence/

[16] - https://profs.info.uaic.ro/~pmihaela/MOC/trajectory.html

[17] - http://www.rasfoiesc.com/educatie/informatica/Particle-Swarm-Optimization-Mo24.php

[18] - http://www.rasfoiesc.com/educatie/informatica/Particle-Swarm-Optimization-Mo24.php

[19] - https://profs.info.uaic.ro/~pmihaela/MOC/PSO.html

[20] - https://www.researchgate.net/figure/Flow-diagram-illustrating-the-particle-swarm-optimization-algorithm\textunderscore fig1\textunderscore 309430266 % Results and Discussion

%\input{Chapters/Chapter7} % Conclusion

%% ----------------------------------------------------------------
% Now begin the Appendices, including them as separate files

\addtocontents{toc}{\vspace{2em}} % Add a gap in the Contents, for aesthetics

\appendix % Cue to tell LaTeX that the following 'chapters' are Appendices

%\input{Appendices/AppendixA}	% Appendix Title

%\input{Appendices/AppendixB} % Appendix Title

%\input{Appendices/AppendixC} % Appendix Title

\addtocontents{toc}{\vspace{2em}}  % Add a gap in the Contents, for aesthetics
\backmatter

%% ----------------------------------------------------------------
\label{Bibliography}
\lhead{\emph{Bibliography}}  % Change the left side page header to "Bibliography"
\bibliographystyle{unsrtnat}  % Use the "unsrtnat" BibTeX style for formatting the Bibliography
\bibliography{Bibliography}  % The references (bibliography) information are stored in the file named "Bibliography.bib"

\end{document}  % The End
%% ----------------------------------------------------------------